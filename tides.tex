The daily mean values extracted from NorKyst800 are viewed as being crudely de-tided. To get tides into the FjordOs CL model, tidal elevation and tidal (barotropic) currents have to be specified separately and superimposed on the daily mean NorKyst800 input. 
%%%%%%%%%%%%%%%%%%%%%%%%%%% Tidal componetns %%%%%%%%%%%%%%%%%%%%%%%%%%%%%%%%%%%%%%
%\clearpage
\begin{table}[t]
 \caption{Simulated tidal amplitudes [cm] and phases [deg] for a location close to the Viker tidal gauge station. The column "ATPXO" refers to the simulated tides using the TPXO Atlantic data base as input, while the column "Modified" refers to a run using adjusted tides as input. The column "Observed" refers to the observed tides at the nearby Viker tidal gauge station and is added for comparison.}
 \label{tab:1}
 \centering
 \newcolumntype{+}{D{+}{\pm}{2}}
% \begin{tabular}{|c|r|++|++|++|} 
 \begin{tabular}{cr++++++} 
  \hline
  Con-    & \mc{1}{c}{Period}      & \mc{2}{c}{Observed}       & \mc{2}{c}{ATPXO}           & \mc{2}{c}{Modified}\\
  stit.   & \mc{1}{c}{[hrs]} 
                    & \mc{1}{c}{[cm]} 
                                   & \mc{1}{c}{[deg]} 
                                                & \mc{1}{c}{[cm]} 
                                                               & \mc{1}{c}{[deg]} 
                                                                             & \mc{1}{c}{[cm]} 
                                                                                            & \mc{1}{c}{[deg]}\\
  \hline
  M$_2$  & 12.4206 & 12.4+0.7 & 115+3  &  9.7+1.1 & 122+6   & 11.8+0.3 & 105+1  \\
  N$_2$  & 12.6583 &  2.8+0.7 &  69+14 &  5.7+1.1 &  81+11  &  3.1+0.3 &  69+5  \\
  S$_2$  & 12.0000 &  2.3+0.7 &  48+15 &  5.1+1.0 &  81+11  &  3.2+0.3 &  67+5  \\
  O$_1$  & 25.8193 &  2.1+0.7 & 282+20 &  3.7+0.4 &  19+8   &  2.9+0.2 & 338+3  \\
  M$_4$  &  6.2103 &  1.4+0.2 & 287+7  &  0.7+0.2 &  25+17  &  1.1+0.0 & 354+1  \\
  Q$_1$  & 26.8684 &  1.0+0.6 & 221+42 &  0.1+0.3 & 215+154 &  0.1+0.1 & 253+156\\
  K$_1$  & 23.9345 &  0.7+0.6 &  98+49 &  1.2+0.5 & 212+23  &  0.1+0.1 & 198+97 \\
  MN$_4$ &  6.2692 &  0.4+0.2 & 270+24 &  1.0+0.2 & 141+12  &  0.3+0.0 &   7+3  \\
  MS$_4$ &  6.1033 &  0.4+0.2 &   5+28 &  1.1+0.2 & 111+12  &  0.6+0.0 &  80+1  \\
   \hline
 \end{tabular}
\end{table}




The tidal input in terms of tidal elevations and currents are based on the TPXO Atlantic database \citep[][hereafter ATPXO]{egber:erofe:2002}\footnote{\texttt{http://volkov.oce.orst.edu/tides/AO.html}}. Before supplying them to the FjordOs CL model they were first modified using measured tides at the Viker tidal gauge station located close to the southern boundary in the Hvaler Archipelago. Included are the nine tidal constituents listed in Tables \ref{tab:1} and \ref{tab:2}. As is evident semi-diurnal constituent M$_2$ is by far the most dominant one, but also N$_2$ and S$_2$ contributes. 
\begin{table}[t]
%\vspace{-1.5cm}
  \caption{As Table \ref{tab:1}, but for the Oscarsborg tidal gauge station.}
  \label{tab:2}
  \centering
 \newcolumntype{+}{D{+}{\pm}{2}}
 \begin{tabular}{cr++++++} 
  \hline
  Con-    & \mc{1}{c}{Period}      & \mc{2}{c}{Observed}       & \mc{2}{c}{ATPXO}           & \mc{2}{c}{Modified}\\
  stit.   & \mc{1}{c}{[hrs]} 
                    & \mc{1}{c}{[cm]} 
                                   & \mc{1}{c}{[deg]} 
                                                & \mc{1}{c}{[cm]} 
                                                               & \mc{1}{c}{[deg]} 
                                                                             & \mc{1}{c}{[cm]} 
                                                                                            & \mc{1}{c}{[deg]}\\
  \hline
  M$_2$   & 12.4206 & 14.1+0.7 & 132+3  & 11.1+1.2 & 128+7   & 13.7+0.3 & 111+2      \\
  N$_2$   & 12.6583 &  3.0+0.8 &  85+15 &  6.6+1.4 &  86+10  &  3.6+0.4 &  75+6      \\
  S$_2 $  & 12.0000 &  2.7+0.8 &  70+18 &  6.1+1.3 &  85+11  &  3.7+0.4 &  70+7      \\
  O$_1$   & 25.8193 &  2.1+0.7 & 286+17 &  3.9+0.5 &  21+8   &  3.1+0.2 & 340+4      \\
  M$_4$   &  6.2103 &  2.1+0.3 & 332+8  &  1.4+0.4 &  44+19  &  2.0+0.0 &  14+1      \\
  Q$_1$   & 26.8684 &  1.0+0.7 & 230+36 &  0.2+0.4 & 204+126 &  0.0+0.2 & 190+165    \\
  K$_1$   & 23.9345 &  1.2+0.5 & 101+35 &  1.1+0.5 & 213+27  &  0.1+0.2 &  44+79     \\
  MN$_4$  &  6.2692 &  0.6+0.3 & 316+26 &  2.0+0.4 & 163+14  &  0.5+0.0 &  29+3      \\
  MS$_4$  &  6.1033 &  0.5+0.3 &  57+32 &  2.2+0.4 & 135+11  &  1.3+0.0 & 106+1      \\
  \hline
 \end{tabular}
\end{table}





The rationale for the modification of the tidal input is that the resolution of the ATPXO, which is 1/30$^{\textrm{o}}$, is too coarse to get the exact phase and amplitude of the tides in Skagerrak correct. To modify the tides we first imposed the nine constituents on the open boundary of tides from the ATPXO database, and let it run for more than a year (the actual period was 12:00 UTC, April 1, 2014 - 12:00 UTC, September 28, 2014). Time series of water level from a location near the Viker and Oscarsborg tidal gauge stations were then extracted and analyzed based on the T\_Tide package described by \cite{pavlo:etal:2002}. The results are shown under column ``ATPXO'' in Tables \ref{tab:1} and \ref{tab:2}. For comparison we have also extracted and analyzed the observed time series from the Viker and Oscarsborg tidal gauge stations compiled from the Norwegian Coastal Administration (Tidevannstabeller for den norske kyst med Svalbard, 2008). The result of this analysis is shown in column ``Observed'', and clearly show that the simulated tides off the mark.  

To better match the observations tidal amplitudes and corresponding phases at the Viker tidal gauge station were then modified by computing an amplitude factor, $c^{(n)}$, and a phaseshift, $\Delta\phi^{(n)}$ for each tidal component $n$ according to:
\beq
  c^{(n)} &=& a^{(n)}_{obs} / a^{(n)}_{sim} \\
  \Delta \phi^{(n)} &=& \phi^{(n)}_{obs} - \phi^{(n)}_{sim}
\eeq
where $a^{(n)}$ is the amplitude and $\phi^{(n)}$ is the phase for tidal component number $n$ of the water level in the column ``ATPXO''. New amplitudes and phases at the boundary were then calculated using the computed amplification factor and phaseshift on both water level and velocity. The modified tides were then supplied to the FjordOs CL and run for a year with tidal forcing only. The new results were then analyzed exactly as for the ATPXO run. The resulting new tidal amplitudes and periods for the locations close to the Viker and Oscarsborg tidal gauge stations are shown in Tables \ref{tab:1} and \ref{tab:2} in column ``Modified''. The results are clearly improved at both stations. In particular we are pleased with the results close to the Oscarsborg tidal gauge station which may be viewed as a control station in that it is far away from the southern boundary where the tidal forcing is imposed.  

