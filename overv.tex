\section{Model description}
\label{sec:overv}
The FjordOs CL model is based on the canonical Regional Ocean Modeling System (ROMS). For a complete description, the reader is referred to  \cite{haidv:etal:2008} and \cite{shche:mcwil:2003, shche:mcwil:2005, shche:mcwil:2009}. 

\subsection{Model overview}
ROMS is a split–explicit, free-surface and terrain-following vertical coordinate oceanic model, where short time steps are used to advance the surface elevation and barotropic momentum equation and where a much larger time step is used for temperature, salinity, and baroclinic momentum. ROMS employs a two-way time-averaging procedure for the barotropic mode which satisfies the 3D continuity equation. A fourth-order, centered advection scheme for momentum and tracers is employed which nessecitate the application of explicit lateral eddy viscosity and diffusion. A K-profile parameterization (KPP) boundary layer scheme \cite{large:etal:1994} parameterizes the subgrid-scale vertical mixing processes. 

\subsection{Configuration for the Oslofjord}
The model .... 

Noen stikkord:
\begin{enumerate} 
 \item Modellområdets geometri og topografi med bilde
 \item Boundaries and boundary conditions
 \item Antall horisontale gitterruter og antall vertikale lag
 \item Fordelingen av de horisontale gitterrutene (oppløsning), gjerne med zoom in på ulike deler av fjorden for å forklare sjeteen (``the jetty'')
 \item Elve-, tidevanns- og atmosfærisk pådrag
 \item Hvilken computer som er brukt (inkl. antall noder)
\end{enumerate} 

